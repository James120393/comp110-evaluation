% Please do not change the document class
\documentclass{scrartcl}

% Please do not change these packages
\usepackage[hidelinks]{hyperref}
\usepackage[none]{hyphenat}
\usepackage{setspace}
\usepackage{graphicx}
\graphicspath{ {C:/Users/james/Documents/GitHub/comp150-desktop-game-Eval/Figures/} }
\doublespace

% You may add additional packages here
\usepackage{amsmath}

% Please include a clear, concise, and descriptive title
\title{Evaluation}

% Please do not change the subtitle
\subtitle{COMP110 - Evaluation}

% Please put your student number in the author field
\author{1506530}

\begin{document}

\maketitle
Over this semester I feel as though I have improved a great deal in both my programming skills and design skills, and feel much more confident in my ability to contribute in the following years. I have chosen a few key skills that I feel I should attempt to improve on over the coming weeks.

\section{Project Tracking}
During my time in semester two I have found that I lacked tracking for my projects.
For  example  my  worksheets  were  not  all  handed  in  on  time  and  one  was  incomplete.
This led to a lack of understanding when interpreting certain programming issues.  A
suggested improvement to this would be to track the projects over the course of the year
and mark the milestones of the projects, i.e.  sprint reviews and deadlines. From this I
can concept a weekly schedule to ensure my progress is constant. To do this I will create a Trello board to more efficiently track my progress for the overall semester and another for the projects themselves.

\section{C++ Language Constructs}
An Issue I ran into was understanding the language constructs, in C++ for example I
struggled to get to grips with the word `Polymorphism', also `Vectors'.  I found that these
two words alone caused a lot of strife as they seemed to either have multiple meanings
or are very descriptive of their use.  As I begin to work on some personal projects I will have to use some of these constructs, and be forced to research
and repeat the use of these words in my programming. This should further my knowledge and skill in the C++ constructs.

/section(UML)
UML I found to be a very useful design tool, I enjoyed making them and using them
to convey my ideas to other people.  The only issue I has with UML was the sequence
diagrams.  I could not manage to understand the  ow of the diagrams, this caused me
to fail in creating one. In the future I shall attempt to create several UML Sequence
diagrams for my group next year, as they will not be as adept in programming it will
be a very useful toll to use.

\section{Final Remarks}
As a next step I believe that a better management projects will significantly improve
my  work  output  and  quality,  also  over the coming weeks  more  practice  and  research  in  UML  and
programming language constructs will help ensure my success in the future semesters. 


\bibliographystyle{ieeetran}
\bibliography{COMP150-Group_Eval}

\end{document}